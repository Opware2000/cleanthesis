% !TEX root = ../thesis-example.tex
%
\chapter{De la caractéristique physique à l'information génétique}
\label{sec:chap1}

\cleanchapterquote{Il y a toujours quelques individus que le hasard isole, ou que la génétique favorise.}{Ronald Wright}{(Chronique des jours à venir)}

\section*{Introduction}
Rappels des classes précédentes :
\begin{itemize}
\item
  Tous les êtres vivants sont formés de cellules.
\item
  Les cellules se composent d'une membrane, d'un cytoplasme et d'un
  noyau.
\item
  Nous appartenons tous à l'espèce humaine (\emph{Homo sapiens}) mais
  nous sommes tous différents.
\end{itemize}

Activité 1 - rappels.

Important~: Deux êtres vivants appartiennent à la même espèce si :
\begin{itemize}
\item
  ils se ressemblent et présentent de nombreux caractères en commun
\item
  ils peuvent se reproduire entre eux
\item
  leurs petits peuvent se reproduire à leur tour
\end{itemize}

\textbf{\emph{Pbm~: Comment expliquer les différences et les
ressemblances qui existent entre les individus~?}}


\section{Les caractéristiques physiques de chaque individu.}
\label{sec:chap1:caracteristiquesphysiques}

\subsection{Activité 2~: Les caractéristiques physiques de chaque individu}
Objectif : Rechercher, extraire et organiser l'information utile (I)

\subsection{Conclusion}

\begin{itemize}
\item
  Chaque être humain possède des caractères en commun (un visage, des
  cheveux\ldots{} caractère de l'espèce = \textbf{caractères
  spécifiques}) et des \textbf{variations} qui lui sont propres (couleur
  des yeux, couleur des cheveux).
\item
  Les caractères que l'on retrouve dans les générations suivantes sont
  des \textbf{caractères héréditaires}.
\item
  Les \textbf{facteurs environnementaux}, conditions de vie (exposition
  au soleil, mode d'alimentation\ldots{}) peuvent modifier certains
  caractères, ces modifications ne sont pas héréditaires.
\end{itemize}

\section{La localisation de l'information à l'origine des caractères
héréditaires.}
\label{sec:chap1:localisation}

\subsection{Activité 3~: Les caractéristiques physiques de chaque individu}
Objectifs : Raisonner, argumenter, pratiquer une démarche expérimentale

\subsection{Conclusion}

Le \textbf{programme génétique} de l'individu se situe dans le
\textbf{noyau} de la \textbf{cellule œuf} (\textbf{zygote}), à l'origine
de toutes les cellules du corps, puis dans le noyau de \textbf{toutes}
les cellules de l'organisme.

\subsection{Activité 4~: L'organisation et le support de l'information héréditaire
dans la cellule.}
Objectifs : rechercher, extraire et organiser de l'information utile,
(I)\\
Communiquer à l'aide d'un dessin d'observation (C).

\subsection{Conclusion}
\begin{itemize}
\item
  Dans le noyau des cellules, on observe des filaments noirs, épais très
  colorables : ce sont des \textbf{chromosomes}.
\item
  \textbf{Toujours présents}, ils sont plus facilement observables
  lorsque la cellule se multiplie/divise.
\item
  Ils portent \textbf{l'information génétique (= héréditaire)}.
\end{itemize}

\section{L'organisation des chromosomes dans la cellule
humaine}

\label{sec:chap1:organisationchromosomes}
\textbf{\emph{Pb : Comment les chromosomes sont-ils organisés dans le
noyau ?}}

\subsection{Activité 5 L'organisation des chromosomes dans la cellule}
Objectifs : rechercher, extraire et organiser de l'information utile (I)

\subsection{Conclusion}

Chaque cellule d'un être humain possède 23 paires (=46) de chromosomes.

On appelle caryotype l'ensemble des chromosomes classés (en fonction de
leur taille, de la longueur de leur bras, et de leurs bandes de
coloration), observés dans une cellule en division.

Tous les individus de la même espèce possèdent le même nombre de
chromosomes et le même caryotype.

La 23ème paire présente des caractéristiques différentes selon le sexe :
ce sont les chromosomes sexuels, ils sont responsables des différences
entre hommes et femmes. Ils sont notés :
\begin{itemize}
\item
  X et X pour la femme ;
\item
  X et Y pour l'homme.
\end{itemize}

Un nombre anormal de chromosomes (= anomalies chromosomiques) empêche le
bon développement de l'embryon et/ou entraîne des caractères différents
chez l'individu concerné : malformations (mentales, physiques) par ex.
la trisomie 21.

