% !TEX root = ../thesis-example.tex
%
\pdfbookmark[0]{Résumé}{Résumé}
\chapter*{Résumé}
\label{sec:abstract}
\vspace*{-10mm}

Cours de \emph{Sciences de la Vie et de la Terre}, dispensés en classe de 3\up{e} durant l'année scolaire 2015-2016, au Collège Pierre \textsc{Larousse}, Toucy.

Le programme est organisé en quatre parties :
\begin{itemize}
\item \textbf{Diversité et unité des êtres humains (30\%)}\\
	A un niveau adapté à la classe de troisième, la notion de programme
génétique permet une première explication de l'unité de l'espèce et de
l'unicité de chaque être humain. Il s'agit d'expliquer de la manière la
plus simple et la plus concrète possible :
	\begin{itemize}
		\item l'influence des facteurs environnementaux sur l'expression des
caractères individuels à travers un ou deux exemples ;
		\item la relation entre information génétique et chromosomes ;
		\item l'existence d'une information génétique (acide désoxyribonucléique
ou ADN) considérée comme identique dans toutes les cellules
somatiques de l'organisme ;
		\item la transmission de l'information génétique ;
		\item l’origine de la diversité des êtres humains. 
	\end{itemize}
\item \textbf{Évolution des êtres vivants et histoire de la Terre (20\%)}\\
La mise en évidence de l'origine des roches sédimentaires, la
reconstitution d'un paysage ancien ont déjà introduit l'idée d'un lien
entre l'histoire de la Terre et celle de la vie et l’idée de changements
au cours des temps. L'étude de quelques exemples significatifs doit
notamment permettre :	
	\begin{itemize}
		\item d’atteindre un premier niveau de formulation de la théorie de
l’évolution des organismes vivants au cours des temps géologiques
présentée sous la forme d’un arbre unique ;
		\item de donner un aperçu de la théorie expliquant ces faits : variation
aléatoire due aux mécanismes de l’hérédité puis sélection par le
milieu des formes les plus adaptées ;
		\item d’aboutir à la recherche d'une explication au niveau génétique par
le réinvestissement des acquis de la partie Diversité et unité des êtres
humains ; 
		\item d’aborder le problème des crises de la biodiversité et de leurs
causes supposées ;
		\item de montrer que la classification scientifique actuelle se fonde sur la
théorie de l’évolution. 
	\end{itemize}
\item \textbf{Risque infectieux et protection de l'organisme (25\%)}\\
Cette partie du programme conduit les élèves à un premier niveau
de compréhension des réactions qui permettent à l'organisme de se
préserver des microorganismes provenant de son environnement.
Il s'agit :
	\begin{itemize}
	\item d'expliquer, à partir de l'analyse de situations courantes, comment
l'organisme réagit à la contamination ;
\item de montrer que l’activité du système immunitaire est permanente
et très souvent efficace vis-à-vis d’une contamination ;
\item de montrer que le fonctionnement du système immunitaire peut
être perturbé (SIDA, allergies, …). 
	\end{itemize}
\item \textbf{Responsabilité humaine en matière de santé et d’environnement (25\%)} \\
Il s’agit :
\begin{itemize}
	\item d’acquérir de nouvelles connaissances et de mobiliser celles
acquises tout au long de la scolarité;
	\item de relier les notions scientifiques et techniques à leurs incidences
humaines en matière de santé et d’environnement ;
	\item de mettre à profit l’attitude d’esprit curieux et ouvert, développée
dans les classes précédentes ;
	\item de travailler les méthodes de raisonnement préservant le libre
arbitre de chacun ;
	\item de développer l’autonomie de l’élève dans une démarche de
projet ;
	\item de permettre aux élèves d’argumenter à partir de bases
scientifiques sur différents thèmes de société.
\end{itemize}
Du point de vue de la responsabilité individuelle et collective on
aborde des questions relatives à l’éducation à la santé et au
développement durable dans les sujets suivants :
\begin{itemize}
	\item les maladies nutritionnelles et certains cancers ;
	\item les transplantations (les dons d'organes, de tissus et de cellules) ;
	\item la qualité de l'eau ou de l'air de la basse atmosphère ;
	\item la biodiversité ;
	\item les ressources en énergies fossiles et énergies renouvelables ;
	\item la maîtrise de la reproduction.
\end{itemize}

Cette partie sera l'occasion d’un croisement des disciplines, d’un
travail au centre de documentation et d’information avec le
professeur documentaliste et, dans la mesure du possible, d’une
collaboration avec des partenaires extérieurs. Pour les projets
consacrés à l’environnement, on veillera à ce qu’ils soient appuyés
sur des exemples pris dans le territoire de l’élève ; traiter de
questions locales d’environnement dans une perspective de
développement durable amène naturellement à ouvrir
l’établissement via les partenariats, à favoriser une implication et
un engagement plus direct des élèves.

Chaque élève, seul ou en groupe, s’implique selon une démarche
de projet dans un sujet. Ce travail aboutit à une production
exploitable collectivement et pouvant intégrer l’usage des
technologies de l’information et de la communication.

L’enseignant encadre le travail des élèves dans toutes les étapes de
la démarche de projet.

L’ensemble des travaux de la classe sera l’objet d’une
mutualisation.
\end{itemize}

