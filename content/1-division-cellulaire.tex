% !TEX root = ../Cours_1S_v0.0.1.tex
%
\chapter{BILAN : L'énergie solaire, la biomasse et les combustibles fossiles}

Résumé~: Le Soleil est une source d'énergie considérable mais
inégalement répartie sur le globe. Les hydrocarbures sont générés par un
processus biologique et chimique qui nécessite des millions d'années. Le
cycle du carbone crée un équilibre entre le carbone minéral et
biologique. Leur consommation rapide impacte de manière importante le
cycle du carbone en augmentant fortement et rapidement la concentration
de CO\textsubscript{2} dans l'atmosphère.


  \section{Le soleil, une source d'énergie
  indispensable}\label{le-soleil-une-source-duxe9nergie-indispensable}


    \subsection{Influence de l'énergie solaire sur la
    Terre}\label{influence-de-luxe9nergie-solaire-sur-la-terre}
  

La planète Terre reçoit un \textbf{rayonnement solaire} qui fournit une
énergie considérable. Cette énergie est sous forme \textbf{lumineuse} et
\textbf{thermique}. Elle a un effet sur les êtres vivants ainsi que sur
des phénomènes planétaires globaux. En cela, \textbf{le Soleil est
indirectement responsable de quasiment toute l'énergie disponible sur
Terre}.

\subsubsection{Répartition de l'énergie solaire sur la
Terre}\label{ruxe9partition-de-luxe9nergie-solaire-sur-la-terre}

L'énergie solaire reçue par la Terre varie en fonction de la
\textbf{latitude}. Elle est maximale à l'équateur, puis elle diminue en
moyenne de l'équateur vers les pôles. En effet, la Terre étant
sphérique, les rayons solaires incidents, qui arrivent parallèlement les
uns aux autres à la surface de la Terre, se répartissent sur une plus
petite surface au niveau de l'équateur qu'au niveau des pôles. Le flux
solaire est donc plus faible lorsque la latitude est grande.

\subparagraph{Flux solaire}\label{flux-solaire}

\subparagraph{\texorpdfstring{Le flux solaire est l'énergie solaire
reçue par la Terre sur une surface donnée. Elle s'exprime donc en
W.m\textsuperscript{−2}.}{Le flux solaire est l'énergie solaire reçue par la Terre sur une surface donnée. Elle s'exprime donc en W.m−2.}}\label{le-flux-solaire-est-luxe9nergie-solaire-reuxe7ue-par-la-terre-sur-une-surface-donnuxe9e.-elle-sexprime-donc-en-w.m2.}

\subparagraph{\texorpdfstring{Le flux solaire moyen au sol est de 200
W.m\textsuperscript{−2}.}{Le flux solaire moyen au sol est de 200 W.m−2.}}\label{le-flux-solaire-moyen-au-sol-est-de-200-w.m2.}

L'énergie solaire est donc inégalement répartie à la surface du globe.
Cela explique notamment les différences de température de l'eau et de
l'air observées entre les régions tropicales et les régions polaires.

\includegraphics[width=5.99701in,height=2.99701in]{media/image1.png}

\protect\hypertarget{ruxe9partition-du-rayonnement-solaire-sur-l}{}{}Figure
1 : Répartition du rayonnement solaire sur le globe terrestre

Comme la Terre réémet une partie de l'énergie reçue, l'énergie
disponible est plus faible que l'énergie totale reçue. Le flux solaire
étant plus important à l'équateur qu'aux pôles, le \textbf{bilan
énergétique} est positif à l'équateur et négatif aux pôles.

\subparagraph{\texorpdfstring{\textbf{Bilan
énergétique}}{Bilan énergétique}}\label{bilan-uxe9nerguxe9tique}

\subparagraph{Le bilan énergétique, ou bilan radiatif est la différence
entre l'énergie absorbée par la Terre et l'énergie qu'elle réémet sous
forme d'infrarouge vers
l'espace.}\label{le-bilan-uxe9nerguxe9tique-ou-bilan-radiatif-est-la-diffuxe9rence-entre-luxe9nergie-absorbuxe9e-par-la-terre-et-luxe9nergie-quelle-ruxe9uxe9met-sous-forme-dinfrarouge-vers-lespace.}

\subparagraph{Le bilan énergétique est positif lorsque la surface
considérée absorbe plus d'énergie qu'elle n'en réémet. A l'inverse le
bilan énergétique est négatif lorsque la surface considérée absorbe
moins d'énergie qu'elle n'en
réémet.}\label{le-bilan-uxe9nerguxe9tique-est-positif-lorsque-la-surface-considuxe9ruxe9e-absorbe-plus-duxe9nergie-quelle-nen-ruxe9uxe9met.-a-linverse-le-bilan-uxe9nerguxe9tique-est-nuxe9gatif-lorsque-la-surface-considuxe9ruxe9e-absorbe-moins-duxe9nergie-quelle-nen-ruxe9uxe9met.}

\subparagraph{Entre l'équateur et jusqu'à environ 35° de latitude le
bilan énergétique est positif. Entre 35° et les pôles à l'inverse, le
bilan énergétique est
négatif.}\label{entre-luxe9quateur-et-jusquuxe0-environ-35-de-latitude-le-bilan-uxe9nerguxe9tique-est-positif.-entre-35-et-les-puxf4les-uxe0-linverse-le-bilan-uxe9nerguxe9tique-est-nuxe9gatif.}

\subsubsection{Conséquence sur les phénomènes de circulation des fluides
de la
planète}\label{consuxe9quence-sur-les-phuxe9nomuxe8nes-de-circulation-des-fluides-de-la-planuxe8te}

Les différences de température provoquées par la répartition inégale de
l'énergie solaire à la surface du globe engendrent des mouvements à
l'échelle du globe. En particulier, se produisent des
\textbf{déplacements atmosphériques et océaniques : le vent et les
courants.}

Dans l'atmosphère, les masses d'air sont animées par des mouvements
horizontaux, engendrés par des différences de pression entre les masses
d'air, et verticaux, engendrés par des différences de densité entre
l'air chaud et l'air froid.

La combinaison de ces mouvements horizontaux et verticaux produit des
\textbf{cellules de convection}, ou cellules de Hadley.

\subparagraph{\texorpdfstring{\textbf{Mouvement de
convection}}{Mouvement de convection}}\label{mouvement-de-convection}

\subparagraph{Le mouvement de convection est un déplacement de matière
(d'air ou d'eau ici) lié à des différences de température. L'air chaud
(ou l'eau chaude) monte, tandis que l'air froid (ou l'eau froide)
descend.}\label{le-mouvement-de-convection-est-un-duxe9placement-de-matiuxe8re-dair-ou-deau-ici-liuxe9-uxe0-des-diffuxe9rences-de-tempuxe9rature.-lair-chaud-ou-leau-chaude-monte-tandis-que-lair-froid-ou-leau-froide-descend.}

\subparagraph{Lorsque l'on fait chauffer de l'eau dans une casserole, on
peut remarquer des mouvements de
convection.}\label{lorsque-lon-fait-chauffer-de-leau-dans-une-casserole-on-peut-remarquer-des-mouvements-de-convection.}

\subparagraph{\texorpdfstring{\textbf{Cellule de
convection}}{Cellule de convection}}\label{cellule-de-convection}

\subparagraph{Les cellules de convection (ou cellules de Hadley) sont
des mouvements de convection particuliers. Ces mouvements sont
circulaires.}\label{les-cellules-de-convection-ou-cellules-de-hadley-sont-des-mouvements-de-convection-particuliers.-ces-mouvements-sont-circulaires.}

Dans les océans, les fluides sont animés par des mouvements affectant
les masses d'eaux superficielles et des mouvements affectant les masses
d'eau profondes, dus à des différences de densité entre l'eau chaude et
l'eau froide. Les eaux chaudes étant moins denses, elles remontent vers
la couche superficielle des océans, alors que les eaux froides, plus
denses, plongent vers les zones profondes. \textbf{Le Soleil est donc
essentiel à l'établissement de courants marins.}

De plus, l'énergie solaire est à l'origine de l'évaporation de l'eau,
qui passe de l'état liquide à l'état gazeux. Cette vapeur d'eau forme
des nuages. Lorsque ces nuages se refroidissent, l'eau qu'ils
contiennent passe à l'état liquide (pluie) ou solide (neige). C'est le
cycle de l'eau, lui aussi généré par le rayonnement du Soleil.


  \subsection{Les énergies renouvelables dérivées de l'énergie
  solaire}\label{les-uxe9nergies-renouvelables-duxe9rivuxe9es-de-luxe9nergie-solaire}

 
    \subsubsection{Les sources d'énergie
    renouvelable}\label{les-sources-duxe9nergie-renouvelable}
 
\subparagraph{Energie renouvelable}\label{energie-renouvelable}

\subparagraph{Une énergie renouvelable est une énergie qui n'est pas
épuisable par l'exploitation humaine, car elle se renouvelle
rapidement.}\label{une-uxe9nergie-renouvelable-est-une-uxe9nergie-qui-nest-pas-uxe9puisable-par-lexploitation-humaine-car-elle-se-renouvelle-rapidement.}

\subparagraph{Le photovoltaïque, l'énergie éolienne ou hydraulique sont
des énergies
renouvelables.}\label{le-photovoltauxefque-luxe9nergie-uxe9olienne-ou-hydraulique-sont-des-uxe9nergies-renouvelables.}

\textbf{L'énergie solaire est une énergie renouvelable}. L'Homme sait
capter directement l'énergie solaire pour la convertir pour ses propres
besoins (eau chaude, électricité). De plus, le rayonnement du Soleil est
à l'origine des vents et du cycle de l'eau, et donc d'autres formes
d'énergies :

\begin{itemize}
\item
  \textbf{L'énergie éolienne}, qui est l'exploitation de la force du
  vent.
\item
  \textbf{L'énergie hydraulique}, qui est fournie par les mouvements de
  l'eau tels que la marée, les cours d'eau ou encore les chutes d'eau.
\item
  L'énergie solaire utilisée par les végétaux pour la synthèse de
  matière organique peut également constituer une source d'énergie
  renouvelable, avec les biocarburants.


    \subsubsection{L'utilisation des énergies
    renouvelables}\label{lutilisation-des-uxe9nergies-renouvelables}


L'Homme exploite déjà certaines des énergies renouvelables, notamment
l'énergie hydraulique qu'il utilise grâce à la mise en place de barrages
sur les cours d'eau. Ces barrages produisent de l'électricité.

L'augmentation du prix des combustibles fossiles et les conséquences
environnementales néfastes qu'ils entraînent incitent l'Homme à utiliser
de nouvelles sources d'énergie pour subvenir à ses besoins énergétiques
:

\begin{itemize}
\item
  \textbf{Les panneaux solaires} (ou photovoltaïques) qui permettent de
  capter directement l'énergie lumineuse.
\item
  \textbf{Les éoliennes} qui utilisent une hélice pour produire de
  l'électricité à partir du vent.
\item
  \textbf{Les hydroliennes} qui utilisent une hélice pour produire de
  l'électricité à partir des courants marins.
\end{itemize}

\textbf{Cette utilisation des vents et des mouvements d'eau revient
indirectement à exploiter l'énergie du rayonnement solaire.}

\textbf{Les biocarburants} sont produits à partir d'exploitations
agricoles. L'huile pour le biodiesel est obtenue à partir de colza ou de
tournesol, l'éthanol pour les voitures à essence est obtenu à partir de
blé ou de betterave. Cependant cette production de biocarburants peut
avoir des effets néfastes sur l'environnement, elle rentre en
concurrence avec les productions agricoles pour l'alimentation et
accentue la déforestation pour étendre les surfaces cultivables dans les
pays en voie de développement.

\textbf{L'Union européenne a fixé l'objectif de produire 20\% d'énergie
renouvelable d'ici à 2020. En France, 15\% de l'énergie était d'origine
renouvelable en 2008.}

Les principaux obstacles à l'utilisation des énergies renouvelables sont
le prix de production, le caractère intermittent et la difficulté de
stockage.


  \section{L'utilisation de l'énergie solaire par la biosphère pour
  produire de la biomasse : la
  photosynthèse}\label{lutilisation-de-luxe9nergie-solaire-par-la-biosphuxe8re-pour-produire-de-la-biomasse-la-photosynthuxe8se}

  
    \subsection{La photosynthèse}\label{la-photosynthuxe8se}


Les végétaux \textbf{chlorophylliens} possèdent dans leurs feuilles et
certaines tiges, des cellules contenant des \textbf{chloroplastes}, qui
sont des organites du cytoplasme. Ces chloroplastes contiennent un
pigment vert, \textbf{la chlorophylle}, qui donne aux végétaux leur
couleur verte.

La \textbf{photosynthèse} est une réaction chimique qui forme de la
matière organique à partir de matière minérale et d'énergie lumineuse.\\
La photosynthèse a lieu dans les chloroplastes. La chlorophylle capte
l'énergie lumineuse nécessaire à la photosynthèse, qui est une réaction
nécessitant de l'énergie. Des ions minéraux sont aussi utilisés, comme
les nitrates NO\textsuperscript{−}\textsubscript{3}, les phosphates
PO\textsuperscript{3−}\textsubscript{4,} et le potassium K+.

Les ions minéraux sont des molécules chargées positivement ou
négativement, qui sont présents dans le sol et absorbés par les racines
de la plante, avec l'eau.

L'équation bilan de la photosynthèse est :

\includegraphics[width=4.86458in,height=1.46875in]{media/image2.png}

\protect\hypertarget{equation-bilan-de-la-photosynthuxe8se}{}{}Figure
2~: Equation bilan de la photosynthèse

Le CO\textsubscript{2} nécessaire à cette réaction est prélevé dans
l'air. L'eau et les ions minéraux sont prélevés dans le sol. Le
\textbf{glucose} formé par la photosynthèse peut ensuite être transformé
en autres molécules organiques, comme les acides aminés (constituants
des protéines).

\textbf{La photosynthèse permet donc le stockage de l'énergie solaire
sous forme de matière organique végétale.}

\includegraphics[width=5.83333in,height=4.02099in]{media/image3.png}

\protect\hypertarget{le-fonctionnement-dune-cellule-chlorophy}{}{}Figure
3 Le fonctionnement d'une cellule chlorophyllienne

\subsection{Les rôles de la
photosynthèse}\label{les-ruxf4les-de-la-photosynthuxe8se}

\textbf{La photosynthèse est indispensable à la vie sur Terre}. Elle
produit du dioxygène O\textsubscript{2}, qui est nécessaire à la
respiration de tous les êtres vivants.

La photosynthèse est également une source importante de production de
biomasse. En effet, la plupart des chaînes alimentaires partent d'un
végétal.

\subparagraph{\texorpdfstring{\textbf{Biomasse}}{Biomasse}}\label{biomasse}

\subparagraph{La biomasse est l'ensemble de la matière organique,
végétale et
animale.}\label{la-biomasse-est-lensemble-de-la-matiuxe8re-organique-vuxe9guxe9tale-et-animale.}

\subparagraph{}\label{section}

\subparagraph{Remarque~: Lorsque l'on parle de biomasse comme source
d'énergie, on considère la plupart du temps que c'est sous forme de
combustion.}\label{remarque-lorsque-lon-parle-de-biomasse-comme-source-duxe9nergie-on-considuxe8re-la-plupart-du-temps-que-cest-sous-forme-de-combustion.}

\subparagraph{\texorpdfstring{\textbf{Chaîne
alimentaire}}{Chaîne alimentaire}}\label{chauxeene-alimentaire}

\subparagraph{Une chaîne alimentaire est un ensemble d'êtres vivants
liés par des relations alimentaires (aussi appelées relations
trophiques).}\label{une-chauxeene-alimentaire-est-un-ensemble-duxeatres-vivants-liuxe9s-par-des-relations-alimentaires-aussi-appeluxe9es-relations-trophiques.}

\subparagraph{Fleur insecte oiseau forment une chaîne
alimentaire.}\label{fleur-insecte-oiseau-forment-une-chauxeene-alimentaire.}

\subparagraph{\texorpdfstring{\textbf{Réseau
trophique}}{Réseau trophique}}\label{ruxe9seau-trophique}

\subparagraph{Un réseau trophique est un ensemble de chaînes
alimentaires liées entre
elles.}\label{un-ruxe9seau-trophique-est-un-ensemble-de-chauxeenes-alimentaires-liuxe9es-entre-elles.}

\subparagraph{Dans l'exemple de la définition précédente, l'insecte peut
aussi être mangé par une grenouille. L'ensemble de ces liens
alimentaires forme un réseau
trophique.}\label{dans-lexemple-de-la-duxe9finition-pruxe9cuxe9dente-linsecte-peut-aussi-uxeatre-manguxe9-par-une-grenouille.-lensemble-de-ces-liens-alimentaires-forme-un-ruxe9seau-trophique.}

On dit que les végétaux sont les \textbf{producteurs primaires}, ils
sont capables de synthétiser leur propre matière organique à partir de
molécules minérales. Ils peuvent se nourrir eux-mêmes et sont donc
\textbf{autotrophes}. Les végétaux servent de nourriture à d'autres
organismes, appelés \textbf{producteurs secondaires}. Les animaux sont
obligés de consommer un autre être vivant pour se nourrir, on dit aussi
qu'ils sont \textbf{hétérotrophes}.

Les \textbf{décomposeurs} terminent le cycle : ils se nourrissent de
restes d'êtres vivants, et produisent de la matière minérale. La
photosynthèse permet donc de fournir de la nourriture aux êtres vivants.

\includegraphics[width=5.83333in,height=3.66098in]{media/image4.png}

\protect\hypertarget{fonctionnement-schuxe9matique-dun-ruxe9seau-tr}{}{}Figure
4 Fonctionnement schématique d'un réseau trophique.

La biomasse végétale, produite par la photosynthèse, est également une
source d'énergie renouvelable. Elle est par exemple utilisée pour
l'alimentation, le papier et le chauffage au bois.


  \section{Les différentes sources d'énergie issues de la biomasse : les
  combustibles
  fossiles}\label{les-diffuxe9rentes-sources-duxe9nergie-issues-de-la-biomasse-les-combustibles-fossiles}


    \subsection{La place du carbone et des molécules organiques dans les
    écosystèmes}\label{la-place-du-carbone-et-des-moluxe9cules-organiques-dans-les-uxe9cosystuxe8mes}


Au sein d'un écosystème, la matière organique qui compose les êtres
vivants est, à leur mort, \textbf{consommée par les décomposeurs}
appartenant au réseau trophique de l'écosystème. La matière organique
qu'ils constituent est oxydée par la respiration et redevient, au bout
d'un certain temps, de la \textbf{matière minérale}. Le carbone se
retrouve alors dans les molécules de \textbf{dioxyde de carbone.} Ce
carbone peut être utilisé à nouveau par des végétaux pour fabriquer de
\textbf{la matière organique :} c'est le \textbf{cycle du carbone.}

Le carbone de la matière minérale peut se trouver sous forme gazeuse,
dissoute ou solide dans les roches.

Cependant, dans certaines conditions, \textbf{la matière organique n'est
pas entièrement dégradée et reste dans cet état pendant des millénaires
et même des millions d'années.} Cela se produit principalement dans des
zones géographiques à forte productivité biologique, comme les zones
tropicales, et également dans les zones où la matière organique morte
est recouverte rapidement par une couche imperméable comme de l'argile.

Le carbone organique peut être alors se retrouver stocké dans des roches
sous forme de \textbf{charbon}, de \textbf{gaz naturel} ou de
\textbf{pétrole} : ce sont les \textbf{combustibles fossiles}.

\subparagraph{\texorpdfstring{\textbf{Combustible
fossile}}{Combustible fossile}}\label{combustible-fossile}

\subparagraph{Les combustibles fossiles sont des combustibles qui se
forment beaucoup plus lentement que l'Homme ne les utilise. Ils sont
donc non
renouvelables.}\label{les-combustibles-fossiles-sont-des-combustibles-qui-se-forment-beaucoup-plus-lentement-que-lhomme-ne-les-utilise.-ils-sont-donc-non-renouvelables.}

\subparagraph{Les hydrocarbures comme le charbon, le pétrole et le gaz
naturel sont des combustibles non
renouvelables.}\label{les-hydrocarbures-comme-le-charbon-le-puxe9trole-et-le-gaz-naturel-sont-des-combustibles-non-renouvelables.}

\textbf{Les combustibles fossiles sont issus de matière organique
végétale, formée à partir d'énergie solaire par les producteurs
primaires. Indirectement, les combustibles fossiles sont donc également
issus de l'énergie solaire.}

Le charbon provient de restes de végétaux continentaux (feuilles, débris
de bois, etc.). Ces végétaux sont riches en \textbf{lignine}, une
molécule qui rigidifie les végétaux aériens et riches en carbone. Le
charbon est donc très \textbf{riche en carbone C}.

Le gaz et le pétrole proviennent de plancton (végétaux marins
microscopiques). Ils sont \textbf{riches en hydrogène H et en carbone
C}.

\subsection{La formation des combustibles
fossiles}\label{la-formation-des-combustibles-fossiles}

Les restes végétaux tombent au fond des étendues d'eau ou dans le sol,
on dit qu'ils \textbf{sédimentent}. Ils se mélangent alors à des
sédiments inorganiques comme de l'argile, formant ainsi des boues riches
en matière organique, appelées \textbf{boues sapropéliques}. Ces boues
contiennent des \textbf{bactéries anaérobies} (c'est-à-dire qu'elles
n'ont pas besoin d'oxygène pour vivre) qui \textbf{minéralisent la
matière organique}. Il se produit alors une dégradation biochimique.
Cela conduit à la formation de \textbf{kérogène}.

Les transformations se poursuivent et l'enfouissement continue, de plus
en plus profondément dans les couches sédimentaires, c'est la
\textbf{subsidence} (enfouissement sous le poids des sédiments).
Interviennent alors des mécanismes physiques, tels que
l'\textbf{augmentation de la température et de la pression}, qui
agissent sur ces matières. Les bactéries n'interviennent alors plus. Le
kérogène subit une \textbf{dégradation thermique}, qui le transforme
progressivement en \textbf{combustible fossile}. Les molécules de
kérogène perdent de plus en plus leurs atomes d'oxygène et d'hydrogène.
Elles deviennent alors de plus en plus riches en carbone. Cela conduit à
la formation d'\textbf{hydrocarbures}, qui sont les principaux
constituants du pétrole.

Le lieu de cette transformation est la \textbf{roche-mère} et \textbf{ce
processus peut prendre des millions d'années}.

Des \textbf{gaz} peuvent également se former, provenant en partie du
\textbf{méthane} produit par les bactéries au début de l'enfouissement.
Ils seront ensuite transformés sous l'effet des conditions de pression
et de température lors de la subsidence.

\includegraphics[width=4.39583in,height=2.51562in]{media/image5.png}

\protect\hypertarget{la-formation-des-gaz-naturels-et-du-puxe9tr}{}{}Figure
5 : La formation des gaz naturels et du pétrole.

Après leur formation, le pétrole (liquide) et le gaz naturel ont
tendance à quitter la roche-mère et à remonter vers la surface. Leur
remontée s'arrête s'ils rencontrent une roche imperméable concave. Avec
la roche poreuse et perméable où restent les hydrocarbures, appelée
roche-magasin, cette roche imperméable forme un \textbf{piège à
hydrocarbures}.

Ces différentes roches sont visibles sur une coupe géologique, qui est
une représentation verticale montrant ce qu'il y a en sous-sol.

\includegraphics[width=5.79167in,height=2.51789in]{media/image6.png}

\protect\hypertarget{les-structures-guxe9ologiques-qui-piuxe8gent-l}{}{}Figure
6~: Les structures géologiques qui piègent les hydrocarbures.

Lorsque l'enfouissement se fait au bord des lacs ou des mers, et que les
sédiments sont riches en feuilles, tiges, troncs d'arbres, champignons,
pollen et spores de fougères, les conditions de sédimentations peuvent
conduire à la formation de roches solides carbonées comme \textbf{la
houille}, \textbf{la tourbe}, \textbf{le lignite}. Ce processus est une
\textbf{carbonification}, formant ainsi \textbf{le charbon}.

\section{L'utilisation des combustibles
fossiles}\label{lutilisation-des-combustibles-fossiles}

Les pièges à hydrocarbures sont recherchés par les pétroliers, car ils
constituent des gisements de pétrole et de gaz naturel. Certains de ces
gisements sont terrestres, d'autres sont en mer et sont aussi appelés
offshore.

Lorsqu'un forage est réalisé, la récupération du pétrole peut se faire
de manière naturelle s'il remonte tout seul vers la surface. Un pompage
est parfois nécessaire : on parle de récupération assistée.

Les gisements d'hydrocarbures sont de plus en plus rares ce qui
entraîone l'augmentation du prix des hydrocarbures. Leur exploitation
est une activité économique. Elle a aussi des conséquences
environnementales (pollution, marée noire).

L'Homme, en brûlant les combustibles fossiles (gaz, charbon et pétrole)
fait retourner du carbone organique à l'état minéral, sous forme de
dioxyde de carbone, ce qui enrichit l'atmosphère en CO2

. En effet, la combustion d'un combustible fossile réalise le processus
inverse de la photosynthèse, qui utilise du dioxyde de carbone
atmosphérique. L'homme réalise cette réaction pour produire de l'énergie
sous forme de chaleur, avec de multiples applications : carburant
automobile, chauffage, etc. C'est ce que l'on peut voir avec l'exemple
de la combustion du gaz naturel.

L'équation bilan de la combustion du gaz naturel est :

CH4 + 2O\textsubscript{2} CO\textsubscript{2} + 2H\textsubscript{2}O +
énergie

Ainsi, l'Homme agit sur le \textbf{cycle du carbone} en provoquant
l'enrichissement très rapide de l'atmosphère en CO\textsubscript{2},
alors que la constitution des gisements de combustibles fossiles prend
des millions d'années.

\subparagraph{\texorpdfstring{\textbf{Cycle du
carbone}}{Cycle du carbone}}\label{cycle-du-carbone}

\subparagraph{Le cycle du carbone est l'ensemble des transports et des
transformations des composés carbonés entre les grands réservoirs de la
biosphère (ensemble des êtres vivants) et de la géosphère (lithosphère,
hydrosphère, atmosphère). Comme il implique à la fois la biosphère et la
géosphère, on dit que c'est un cycle
biogéochimique.}\label{le-cycle-du-carbone-est-lensemble-des-transports-et-des-transformations-des-composuxe9s-carbonuxe9s-entre-les-grands-ruxe9servoirs-de-la-biosphuxe8re-ensemble-des-uxeatres-vivants-et-de-la-guxe9osphuxe8re-lithosphuxe8re-hydrosphuxe8re-atmosphuxe8re.-comme-il-implique-uxe0-la-fois-la-biosphuxe8re-et-la-guxe9osphuxe8re-on-dit-que-cest-un-cycle-bioguxe9ochimique.}

\includegraphics[width=5.83333in,height=2.98333in]{media/image7.png}

\protect\hypertarget{le-cycle-naturel-du-carbone}{}{}Figure 7 : Le cycle
naturel du carbone

Parmi les phénomènes impliqués dans ce cycle, la photosynthèse utilise
du CO\textsubscript{2} et le stocke dans la biosphère, tandis que la
respiration en rejette vers l'atmosphère.

L'activité humaine influence ce cycle, en particulier depuis la
Révolution industrielle. L'utilisation de quantités très importantes de
combustibles fossiles a entraîné une émission de gaz et notamment de
dioxyde de carbone. Ces gaz rejoignent l'atmosphère. Les végétaux et les
océans réabsorbent une partie du CO\textsubscript{2} émis, mais la
surface végétale terrestre diminue, notamment du fait de la
déforestation. Par conséquent, \textbf{la concentration atmosphérique de
dioxyde de carbone augmente depuis les années 1850}. Or le
CO\textsubscript{2} est un gaz à effet de serre. De ce fait, ces
modifications rapides du cycle du carbone perturbent très fortement le
climat.

\includegraphics[width=5.83333in,height=2.98333in]{media/image8.png}

\protect\hypertarget{linfluence-de-lactivituxe9-humaine-sur-le-c}{}{}Figure
8 : L'influence de l'activité humaine sur le cycle du carbone.
