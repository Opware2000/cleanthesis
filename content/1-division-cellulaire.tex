% !TEX root = ../Cours_1S_v0.0.1.tex
%
\chapter{La reproduction conforme de la cellule et la réplication de l’ADN}
\label{sec:reproduction_conforme}

\rappels{
Rappels : Les cellules sont les plus petites unités fonctionnelles des êtres vivants. Leur patrimoine génétique est codé par les gènes (séquence de nucléotides) au niveau de l’ADN.\\
Les cellules d’un organisme pluricellulaire sont toutes issues d’une seule cellule œuf, qui s’est divisée un très grand nombre de fois. Les cellules filles, au cours des générations cellulaires, conservent les caractéristiques du caryotype (ex : 2n = 46 chromosomes). \\
La mitose est la division cellulaire qui permet d’obtenir 2 cellules identiques à la cellule mère : c’est une \textbf{reproduction conforme}. 
}

\emph{
Pbm : 	Comment les caractéristiques d’une cellule sont-elles maintenues au cours de sa vie ? \\
Comment la reproduction conforme de la cellule est-elle assurée ?
}

\section{La reproduction cellulaire ou mitose.}
\label{sec:mitose}

Au cours de sa vie, une cellule se trouve : 
\begin{itemize}
\item soit dans un état de croissance, assurant la préparation de la prochaine division : \emph{l’interphase} ;
\item soit en division, cette période est appelée mitose = \emph{phase M}. 
\end{itemize}
Les chromosomes sont des structures constantes des cellules eucaryotes, présents durant tout le cycle cellulaire. Ils sont visibles au microscope lors de la mitose, mais quasiment pas lors de l’interphase. Ils sont sous forme de chromatine (ADN + protéines) dans des états de condensation variables.
Au début de la mitose, les chromosomes sont constitués de deux chromatides identiques, contenant chacune une molécule d’ADN.
La mitose comprend 4 phases :

Attention !\begin{itemize}
\item	Il faut bien faire la distinction entre deux chromosomes homologues et les deux chromatides d'un même chromosome 
\item	les chromosomes homologues, qu'ils soient simples ou doubles, possèdent les mêmes gènes, mais possèdent des allèles différents pour certains de ces gènes ;
\item	les deux chromatides d'un chromosome sont génétiquement identiques : elles possèdent les mêmes allèles des mêmes gènes.\end{itemize}
\section{Duplication de l'ADN, une étape fondamentale précédent la mitose (reproduction conforme)}
\subsection{Évolution de la quantité d'ADN au cours du cycle cellulaire}
Le cycle cellulaire est la succession d'une mitose (phase M) et d'une interphase (comprenant les phases G1, S et G2). Durant le cycle cellulaire, l'organisation et la condensation des chromosomes varient de chromatide simple (en filament « | ») à chromatides doubles (chromosome en X).
Durant la phase S (synthèse), la quantité d'ADN est multipliée par 2. A l’issue de cette phase, chaque gène se trouve dupliqué en deux copies présentes sur chacune des chromatides des chromosomes. Ainsi, les 2 chromatides sœurs contiennent la même copie de l'information génétique. C'est la réplication de l’ADN. 
La réplication permet le passage des chromosomes simples (à une chromatide) à des chromosomes doubles (à deux chromatides sœurs).
La phase S est une phase de synthèse, tandis que les phases G1 et G2 sont des phases de croissance cellulaire. 
\subsection{Réplication de l'ADN}
Durant l'interphase, on observe sur l'ADN, des « yeux » de réplication qui vont s'allonger. C'est à leur niveau que se forment les 2 chromatides.



